% !TEX TS-program = pdflatex
% !TEX encoding = UTF-8
% !BIB program = biber

\documentclass[12pt,a4paper]{report}

% generira bookmarkove
\usepackage{bookmark}

% Podrška za hrvatski
\usepackage[croatian]{babel}
\usepackage{csquotes}

% Margine
\usepackage[left=3cm,top=2.5cm,right=2.5cm,bottom=3.5cm]{geometry}

% Simboli i naredbe za matematiku
\usepackage{amsmath, amsthm, amssymb}
\usepackage{mathtools}
\usepackage{subcaption} % side-by-side grafovi

% Podrška za slike
\usepackage{graphicx} % \includegraphics [k=v] interface

% Poveznice
\usepackage{hyperref}

% Blokovi s kodom
\usepackage{minted}
\ifx\write18\relax\else
\usemintedstyle{sas}
\fi

% Literatura
\usepackage[sorting=none,backend=biber,style=numeric]{biblatex}
\addbibresource{seminar.bib}

% Fonts:
\usepackage[T1]{fontenc} % Allows specifying fonts
\usepackage{helvet} % Helvetica; phv
\usepackage{palatino} % Palatino; ppl

\gdef \title{Testiranje}
\gdef \class{Uvod u programsko inženjerstvo}
\gdef \author{Tin Švagelj}
\gdef \uniprogram{Preddimplski studij informatike}
\gdef \semguide{Izv. prof. dr. sc. Sanja Čandrlić}
\gdef \author{Tin Švagelj}
\gdef \date{\today}

% Custom commands and redefinitions
\newcommand{\UseFont}[1]{\fontfamily{#1}\selectfont}

\usepackage{titlesec}
\titleformat{\chapter}[hang]
{\normalfont\Large\bfseries}{\thechapter}{20pt}{\Large}

% \usepackage{fancyhdr}
% \pagestyle{fancy}
% \fancyhead{}
% \renewcommand\headrulewidth{0.1pt}
% \fancyhead[L]{\footnotesize \leftmark}
% \fancyhead[R]{\footnotesize \thepage}
% \renewcommand\headrulewidth{0pt}
% \fancyfoot[R]{\small Fakultet Informatike i\\Digitalnih Tehnologija}
% \renewcommand\footrulewidth{0.1pt}
% \fancyfoot[C]{2022 - 2023}
% \fancyfoot[L]{\small \title}

\begin{document}

\thispagestyle{empty}

\begin{titlepage}
	\begin{center}
        {
            \bf
            FAKULTET INFORMATIKE I\\
            DIGITALNIH TEHNOLOGIJA\\
            \vspace{2pt}\uniprogram
        }

		\vspace*{\stretch{0.5}}
        {\LARGE Dokumentacija iz kolegija}\\
        \vspace{8pt}{\LARGE \class}\\
		\vspace{16pt}{\UseFont{phv} \Huge \bf \title}
        \vspace*{\stretch{0.5}}
	\end{center}

    {
        \renewcommand{\arraystretch}{1.5}
        \begin{tabular}{l l}
            {\bf Autor:} & {\bf \author} \\
            {Mentor/ica:} & {\semguide} \\
        \end{tabular}
    }

    \vspace*{\stretch{0.25}}
	\begin{center}
		{U Rijeci, \today}
	\end{center}
\end{titlepage}


\pagenumbering{roman}

\tableofcontents
\newpage

\pagenumbering{arabic}
\setcounter{page}{1}

\chapter{Testiranje jedinica}

Avalonia UI podržava testiranje jedinica koda (engl. unit test).
Takvi testovi dozvoljavaju provjeru određenih segmenata koda koji nisu usko vezani uz druge ponašanjem (tj. funkcioniraju sami za sebe).

U slučaju Wordel aplikacije je bilo smisleno koristiti ovakvo testiranje za provjeru rada \verb|WordUtil#MatchInput| funkcije koja služi za usporedbu riječi i vraća polje \verb|LetterUse|
koje daje povratnu informaciju koja slova unesene riječi se nalaze u očekivanoj riječi, koja su samo na krivom mjestu, i koja nisu uključena.
S obzirom da funkcija ne ovisi o stanju aplikacije, lagano ju je testirati.

Ti testovi su uvedeni u \textit{TextMatchingTest.cs} datoteci:

\inputminted{cs}{../../Wordel/Wordel.Tests/TextMatchingTest.cs}

Prilikom implementacije testova sam pronašao pogrešku u testiranom algoritmu s \verb|MatchInput_OnePossible_ReturnsWrongsIfCorrectlyUsedLater| testom.

\chapter{Testiranje grafičkog sučelja}

Testiranje grafičkog sučelja (engl. UI test) dozvoljava automatiziranje testiranja interakcije korisnika s grafičkim sučeljem.
S obzirom da je Avalonia UI framework koji cilja nekoliko različitih platformi, provođenje ovakvih testova bi bilo potrebno za svaku od ciljanih platformi.

\smallskip

\textbf{Selenium} dozvoljava testiranje grafičkog sučelja u pregledniku.\par
\textbf{Appium} dozvoljava testiranje grafičkog sučelja za nativne aplikacije.\par
Avalonia UI ima i svoj framework za testiranje grafičkih sučelja koji se zove \textbf{Avalonia.UnitTests}.\par

S obzirom na stupanj kopleksnosti aplikacije, nisam smatrao potrebnim ovu vrstu testova, no u velikim aplikacijama
koje imaju mnogo dinamičnih sučelja bi ovakvi testovi bili iznimno korisni za potvrđivanje da korisnici imaju dobro iskustvo prilikom korištenja aplikacije.


\end{document}
